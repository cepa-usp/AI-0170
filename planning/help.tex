\documentclass[a4paper,10pt]{scrartcl}
  \usepackage[utf8]{inputenc}
  \usepackage[light,math]{kurier}
  \usepackage[T1]{fontenc}
  \usepackage{indentfirst}
  \usepackage{paralist}
  
  \title{Texto de ajuda da AI-0170}
  \author{I. R. Pagnossin @ CEPA}

\begin{document}

\maketitle

\begin{abstract}

\end{abstract}

	Os textos de ajuda devem aparecer numa janela que sobrepõe-se ao plano cartesiano, mas não ao menu (GUI). Alguns deles têm informações que devem ser preenchidas dinamicamente, consultando o objeto que representa o plano cartesiano.

  \section*{Ajuda associada à GUI}
	A barra de menus, que está sempre visível (exceto na tela 0), deve sempre disponibilizar os seguintes textos de ajuda.
Menu principal
	Utilize o menu principal para criar uma nova atividade, para salvá-la ou para abrir uma atividade já salva.


  \subsection*{Tela 0 (novo e abrir)}
    Não há texto de ajuda nesta tela.
	
  \subsection*{Tela 1 (escolha da função)}
    \textbf{Título}: Função $f$

    Escolha a função que deseja integrar.

  \subsection*{Tela 2 (escolha do intervalo de integração)}
    \textbf{Intervalo de integração}

    Os pontos $A$ e $B$ sobre o eixo das abscissas representam os limites inferior e superior de integração, respectivamente. Arraste-os para esquerda ou para a direita para definir o intervalo de integração $[A,B]$ desejado. Você também pode invertê-los de posição, representando assim uma integração no sentido oposto à orientação do eixo das abscissas.

  \subsection*{Tela 3 (escolha da estratégia)}
    \textbf{Estratégia}

    Escolha a estratégia de construção da soma de Riemann que melhor se adequa à sua necessidade:
    \begin{compactdesc}
    \item{Soma inferior:} caso você escolha esta opção, os pontos da partição serão automaticamente definidos
    com base na quantidade $n$ de sub-intervalos, que por sua vez será definida na tela seguinte.
    Além disso, o software automaticamente utilizará o menor valor de $f$ em cada sub-intervalo $[x_i,x_i+\Delta x_i]$ para calcular a soma de Riemann
    (a soma dos elementos de área). Escolha esta opção caso você queira explorar o conceito de soma inferior ou para exibir dinamicamente da soma de Riemann
    em função de $n$ cada vez maior.    
    \item{Soma superior:} caso você escolha esta opção, os pontos da partição serão automaticamente definidos com base na quantidade $n$ de sub-intervalos, que por sua vez será definida na tela seguinte. Além disso, o software automaticamente utilizará o maior valor de f em cada sub-intervalo [xi,xi+xi] para calcular a soma de Riemann (a soma dos elementos de área). Escolha esta opção caso você queira explorar o conceito de soma superior ou para exibir dinamicamente da soma de Riemann em função de n cada vez maior.    
    \item{Soma personalizada:} caso você escolha esta opção, os pontos da partição poderão ser criados livremente, inclusive com amplitudes xi variáveis. Além disso, você terá a liberdade de escolher os pontos de avaliação i de cada sub-intervalo [xi,xi+xi]. Escolha esta opção caso você queira demonstrar paulatinamente a construção genérica da soma de Riemann, a arbitrariedade na escolha da partição ou o Teorema do Valor Médio (TVM) para integrais.
    \end{compactdesc}

  \subsection*{Tela 4 (escolha da partição)}
    \textbf{Título}: Partição
    
    A partição é o conjunto de pontos sobre o eixo das abscissas que, juntamente com os pontos $A$ e $B$, dividem o intervalo de integração em $n$ sub-intervalos.
    
    Caso você tenha escolhido a estratégia ``soma personalizada'' na tela 3, então nesta tela você poderá criar os pontos da partição livremente. Para adicionar um ponto na partição, pressione sobre o eixo $x$ (dentro do intervalo de integração) e, em seguida, pressione o botão $+$ na barra de ferramentas. Para remover um ponto da partição, pressione sobre o ponto desejado para selecioná-lo e, em seguida, pressione o botão $-$ (os limites de integração, que fazem parte da partição, não podem ser removidos).
    
    Caso você tenha escolhido a estratégia ``soma inferior'' ou ``soma superior'', então nesta tela você poderá aumentar ou reduzir a quantidade $n$ de sub-intervalos de $[A,B]$. Neste caso, os pontos da partição serão automaticamente criados de modo que fiquem uniformemente distribuídos ao longo do intervalo de integração (esta restrição não é obrigatória. Aqui ela tem o intuito exclusivo de simplificar a construção da soma de Riemann para $n$ grande). Utilize os botões $+$ e $-$ para aumentar e reduzir $n$, respectivamente.

  \subsection*{Tela 5 (escolha dos elementos de área)}
    \textbf{Título}: Elementos de área
    
    Caso você tenha escolhido a estratégia ``soma personalizada'' na tela 3, então nesta tela você poderá escolher livremente os pontos $\zeta_i$ de avaliação de $f$. Para isso, pressione sobre o gráfico de $f$ na posição $x$ em que deseja criar o ponto de avaliação. Em seguida, pressione o botão $+$ para adicionar o ponto ali. Neste momento, o software automaticamente construirá um elemento de área retangular de base $\Delta x_i$ (amplitude do sub-intervalo) e altura $f_i \equiv f(\zeta_i)$ (valor de $f$ no ponto de avaliação). Repita esse processo para adicionar outros pontos de avaliação (um para cada sub-intervalo) ou para ajustar o ponto de avaliação de um sub-intervalo que já tenha um. Para remover um ponto de avaliação (e o elemento de área correspondente), selecione-o e, em seguida, pressione o botão $-$ na barra de ferramentas.
    
    Caso você tenha escolhido a estratégia ``soma inferior'' ou ``soma superior'', então nesta tela você não precisará fazer nada. Porém, você ainda pode alterar a quantidade $n$ de sub-intervalos, como na tela anterior. Para isso, utilize os botões $+$ e $-$ na barra de ferramentas.

  \subsection*{Tela 6 (Ajuste da constante de integração)}
    \textbf{Teorema Fundamental do Cálculo}
    
    Arraste o gráfico da primitiva de $f$, a função $F$, para cima ou para baixo no plano cartesiano. Este processo representa a escolha de diferentes valores da constante de integração $C$. Um caso particular interessante é aquele em que $C = -F(A)$, o que corresponde a posicionar $F$ de modo que seu gráfico em $x = A$ cruze o eixo das abscissas, isto é, $y = F(A) = 0$. Nesta situação, o valor $y = F(B)$ é precisamente a integral definida de $f$ no intervalo de integração escolhido.
    
    Note que quando $n$ tende ao infinito (estratégias ``somas inferior/superior''), o valor da soma de Riemann, exibida no canto superior direito da tela, aproxima-se de $F(B)-F(A)$. Isto também acontece quando os pontos de avaliação $\zeta_i$ de cada sub-intervalo foram cuidadosamente escolhidos de modo que $f_i$ seja o valor médio de $f$ no sub-intervalo $[x_i,x_i+\Delta x_i]$, independentemente da quantidade $n$ de sub-intervalos (estratégia ``soma personalizada'').

  \section*{Ajuda associada aos EOI}

  \subsection*{$A$ (limite inferior do intervalo de integração)}
    \begin{compactdesc}
      \item{Título:} limite inferior do intervalo de integração
      \item{Símbolo:} $A$
      \item{Valor:} <dinâmico>
      \item{Descrição:} denota o valor da abscissa a partir da qual a integração \emph{começa}. Embora seja comum escolher $A < B$, esta relação não é obrigatória. De fato, $A \ge B$ também é possível. Na soma de Riemann, $A$ é o primeiro elemento da partição do intervalo de integração.
      \item{Interação:} você pode alterar os limites de integração na tela 2.
    \end{compactdesc}

  \subsection*{$B$ (limite superior do intervalo de integração)}
    \begin{compactdesc}
      \item{Título:} limite superior do intervalo de integração
      \item{Símbolo:} $B$
      \item{Valor:} <dinâmico>
      \item{Descrição:} denota o valor da abscissa na qual a integração \emph{termina}. Embora seja comum escolher $A < B$, esta relação não é obrigatória. De fato, $A \ge B$ também é possível. Na soma de Riemann, $B$ é o último elemento da partição do intervalo de integração.
      \item{Interação:} você pode alterar os limites de integração na tela 2.
    \end{compactdesc}

  \subsection{$x_i$ (ponto da partição que não A e B)}
    \begin{compactdesc}
      \item{Título:} $i$-ésimo elemento da partição do intervalo de integração
      \item{Símbolo:} $x_i$
      \item{Valores:} <dinâmico> e $i = $<dinâmico>
      \item{Descrição:}\par
      Na soma de Riemann, uma partição do intervalo de integração é uma divisão desse intervalo em $n$ sub-intervalos menores. Essa divisão é feita escolhendo-se $n+1$ pontos da partição, começando em $x_0 = A$ e terminando em $x_n = B$. Note que existem infinitas partições possíveis (arbitrárias) de um mesmo intervalo de integração.
      \item{Interação:} você pode adicionar e remover pontos da partição na tela 4.
    \end{compactdesc}

  \subsection*{$\zeta_i$ (pontos de avaliação de f)}
	Título: ponto de avaliação de f
	Símbolo: i
	Valor: <dinâmico>
	Descrição: no procedimento da soma de Riemann, escolhemos um ponto i dentro de cada sub-intervalo [xi,xi+xi] no qual o valor da função será calculado: f(i). Este valor representa a altura o elemento de área associado ao sub-intervalo. O valor de i é arbitrário, mas no caso em que é escolhido de tal modo que f(i) é máximo no sub-intervalo, a soma de Riemann é também conhecida por “soma superior”. Analogamente, quando f(i) é mínimo no sub-intervalo, temos a “soma inferior”.
	Interação: você pode adicionar e remover pontos de avaliação na tela 5 caso tenha optado pela estratégia “soma personalizada”, na tela 3. Caso contrário, os pontos de avaliação são definidos automaticamente pelo software (somas inferior e superior).
	
  \subsection*{$f_i\equiv f(\zeta_i)$ (altura do elemento de área)}
	Título: altura do elemento de área
	Símbolo: fif(i)
	Valor: <dinâmico>
	Descrição: no procedimento da soma de Riemann, escolhemos um ponto i dentro de cada sub-intervalo [xi,xi+xi] no qual o valor da função será calculado: f(i), também representado por fi para simplificar. Este valor representa a altura o elemento de área associado ao sub-intervalo. O valor de i é arbitrário, mas no caso em que é escolhido de tal modo que fi é máximo no sub-intervalo, a soma de Riemann é também conhecida por “soma superior”. Analogamente, quando fi é mínimo no sub-intervalo, temos a “soma inferior”.
	Interação: a altura do elemento de área é determinada automaticamente pelo software quando o usuário escolhe o ponto de avaliação i.

  \subsection*{$(\zeta_i,f_i)$}
	Título: nenhum
	Símbolo: (i,fi)
	Valor: <dinâmico, com formato (x.xx, y.yy)>
	Descrição: o ponto geométrico cujas coordenadas são i e fi.
	Interação: você pode adicionar e remover pontos de avaliação (i) na tela 5 caso tenha optado pela estratégia “soma personalizada”, na tela 3. Caso contrário, os pontos de avaliação são definidos automaticamente pelo software (somas inferior e superior). Note que fif(i) é determinado automaticamente a partir de i.

  \subsection*{$\Delta x_i$ (amplitude do i-ésimo sub-intervalo de integração)}
	Título: amplitude do i-ésimo sub-intervalo de integração
	Símbolo: xi
	Valor: <dinâmico>
	Descrição: corresponde à base do elemento de área associado ao sub-intervalo [xi,xi+xi], cuja área é igual a fixi. Note que fif(i), com i[xi,xi+xi].
	Interação: caso você tenha escolhido a estratégia “soma personalizada” (tela 3), a amplitude de cada sub-intervalo de integração dependerá da escolha dos pontos da partição (tela 4). Neste caso, xi terá um valor particular para cada sub-intervalo. Por outro lado, caso você tenha escolhido “soma inferior” ou “soma superior”, xi será automaticamente calculado com base na quantidade nde sub-intervalos (tela 5) por meio da expressão xi=(B-A)/n. Note ainda que, apenas por simplicidade, neste caso xi terá o mesmo valor para todos os sub-intervalos. Caso você queira criar uma soma inferior ou superior com amplitudes distintas, escolha a estratégia “soma personalizada” e escolha manualmente os valores mínimo ou máximo de fi em cada sub-intervalo.

  \subsection*{$\Delta a_i$ (elemento de área)}
	Título: elemento de área
	Símbolo: ai
	Valor: <dinâmico>
	Descrição: representa a parcela de área associada ao sub-intervalo i, que vai de xi até xi+xi. Na construção geométrica da soma de Riemann apresentada aqui, os elementos de área são os retângulos de base xi e altura f(i), cuja área é ai=fixi.
	Interação: os elementos de área são definidos com base nas escolhas da partição (tela 4) e dos pontos de avaliação de f (nesta tela): caso você tenha escolhido a estratégia “soma personalizada”, utilize os botões + e - na barra de ferramentas à esquerda para adicionar ou remover pontos de avaliação de f; caso tenha escolhido “soma inferior” ou “soma superior”, os botões + e - acrescem ou reduzem a quantidade n de sub-intervalos.

  \subsection*{$C$ (constante de integração)}
	Título: constante de integração
	Símbolo: C
	Valor: <dinâmico>
	Descrição: o ponto selecionado é, na verdade, uma representação da constante de integração, um valor constante que quando acrescentado à primitiva F(x) não afeta a sua derivada, isto é, f(x)=d/dx [F(x)+C]. Quando essa constante é ajustada de modo que F(A)=0, o valor F(B) é igual à integral de f no intervalo [A,B].
	Interação: você pode ajustar a constante de integração arrastando o gráfico da primitiva para cima ou para baixo no plano cartesiano, na tela 6.

  \subsection*{$f$ (integrando)}
	Título: integrando
	Símbolo: f
	Valor: <dinâmico: exibir a expressão da função f>
	Descrição: gráfico da função que se quer integrar.
	Interação: você pode escolher a função que será integrada na tela 1.

  \subsection*{$F$ (primitiva)}
	Título: primitiva
	Símbolo: F(x)+C
	Valor: <dinâmico: exibir a expressão da função F>
	Descrição: a primitiva, ou antiderivada, de f é a função F cuja derivada é f. Ou seja, f=dF/dx. Como a derivada de uma constante é zero, podemos adicionar uma constante C arbitrária a F sem alterar a relação anterior. Isto é, f=d/dx [F(x)+C]. Em problemas aplicados a constante de integração é usada para ajustar a solução matemática ao problema que se quer resolver. Por exemplo, na determinação da área sob a curva f, a constante de integração deve ser ajustada de modo que C=-F(A). Nesta condição, o valor F(B)+C=F(B)-F(A) é precisamente a área procurada, ou ainda, a integral definida de f no intervalo [A,B].
	Interação: a primitiva é automaticamente definida ao escolher f, na tela 1, mas a constante de integração pode ser definida arrastando o gráfico de F para cima ou para baixo, na tela 6.

  \subsection*{Soma de Riemann}
	Título: soma de Riemann
	Símbolo: i=0n fixi
	Valor: <dinâmico>
	Descrição: a soma de Riemann é simplesmente a soma das áreas dos retângulos (elementos de área) de base xi e altura fi. Quando n, a área de cada elemento de área tende a zero, enquanto a soma dessas áreas, tende à área sob f. Esta é a integral de Riemann.
	Interação: a soma de Riemann depende da função f escolhida (tela 1), dos limites de integração (tela 2), dos pontos da partição (tela 4) e dos pontos de avaliação de f (tela 5).

  \subsection*{$n$ (quantidade de sub-intervalos)}
	Título: quantidade de sub-intervalos
	Símbolo: n
	Valor: <dinâmico>
	Descrição: n1 é a quantidade de divisões, ou sub-intervalos, do intervalo de integração [A,B].
	Interação: a quantidade de sub-intervalos pode ser controlada na tela 4.

  \subsection*{$F(A)$ (valor da primitiva em $x=A$)}
	Título: valor da primitiva em x=A
	Símbolo: F(A)
	Valor: <dinâmico>
	Descrição: os valores da primitiva em x=A e x=B podem ser utilizados para determinar a integral definida de f no intervalo [A,B], conforme estabelece o Teorema Fundamental do Cálculo (TFC): F(B)-B(A)=ABf(x) dx. Por meio do TFC, podemos evitar calcular a soma de Riemann se conhecermos a primitiva (ou antiderivada) de f, isto é, se conhecermos F.
	Interação: o valor de F(A) depende de f (tela 1) e da constante de integração C (nesta tela), que por sua vez pode ser definida arrastando-se o gráfico de F para cima ou para baixo no plano cartesiano.

  \subsection*{$F(B)$ (valor da primitiva em $x=B$)}
	Título: valor da primitiva em x=B
	Símbolo: F(B)
	Valor: <dinâmico>
	Descrição: os valores da primitiva em x=A e x=B podem ser utilizados para determinar a integral definida de f no intervalo [A,B], conforme estabelece o Teorema Fundamental do Cálculo (TFC): F(B)-B(A)=ABf(x) dx. Por meio do TFC, podemos evitar calcular a soma de Riemann se conhecermos a primitiva (ou antiderivada) de f, isto é, se conhecermos F.
	Interação: o valor de F(B) depende de f (tela 1) e da constante de integração C (nesta tela), que por sua vez pode ser definida arrastando-se o gráfico de F para cima ou para baixo no plano cartesiano.

\end{document}
