\documentclass[a4paper,10pt]{scrartcl}
  \usepackage[utf8]{inputenc}
  %\usepackage[light,math]{kurier}
  \usepackage{arev}
  \usepackage[T1]{fontenc}
  \usepackage[brazil]{babel}
  \usepackage{paralist}

  \setlength{\parindent}{0cm}
  
\begin{document}
    
    \textbf{Símbolo:}\\
      \textbf{Valor:}\\
      \textbf{Descrição:}\\
     \textbf{Interação:}\par\vspace{5mm}
     \large
    \textbf{Função $f$}\\
    \textbf{Intervalo de integração}\\
    \textbf{Estratégia}\\
    \textbf{Partição}\\
    \textbf{Elementos de área}\\
    \textbf{Teorema Fundamental do Cálculo} \\
    \textbf{Adicionar pontos na partição}    \\
    \textbf{Adicionar pontos de avaliação}    \\
    \textbf{Remover pontos da partição}    \\
    \textbf{Remover pontos de avaliação}    \\
    \textbf{Limite inferior do intervalo de integração}\\
    \textbf{Limite superior do intervalo de integração}\\
    \textbf{$i$-ésimo ponto da partição do intervalo de integração}\\
    \textbf{$i$-ésimo ponto de avaliação de $f$}\\
    \textbf{Altura do $i$-ésimo elemento de área}\\
    \textbf{$i$-ésimo ponto de avaliação e valor de $f$ nesse ponto}\\
    \textbf{Amplitude do $i$-ésimo sub-intervalo de integração}\\
    \textbf{Elemento de área}\\
    \textbf{Constante de integração}\\
    \textbf{Integrando}\\
    \textbf{Primitiva}\\
    \textbf{Soma de Riemann}\\
    \textbf{Quantidade de sub-intervalos}\\
    \textbf{Valor da primitiva em $x = a$}\\
    \textbf{Valor da primitiva em $x = b$}
    
\end{document}
