\documentclass[a4paper,12pt]{scrartcl}
  \usepackage[utf8]{inputenc}
  \usepackage[T1]{fontenc}
  \usepackage{gfsartemisia-euler}
  \usepackage[brazil]{babel}
  \usepackage{paralist}
  \usepackage{amsmath}
  \usepackage{amssymb}
  \usepackage{geometry}
  \usepackage{hyperref}
  \usepackage[squaren]{SIunits}
  \usepackage{indentfirst}
  \usepackage{graphicx}

  \newenvironment{howto}{\begin{quotation}\footnotesize\sffamily}{\end{quotation}}
  
  
  \title{Proposta de atividade}
  \subtitle{com o recurso educacional \emph{Integral de Riemann}}
  \author{R. S. Morais, S. N. Dezidério, I. R. Pagnossin}

  \hyphenation{pro-ble-ma pri-mi-ti-va}
  
\begin{document}

  \setlength\parindent{2em}

\maketitle

\begin{abstract}
  Roteiro de aula para a introdução dos conceitos:
  \begin{compactitem}
   \item Soma de Riemann
   \item Teorema fundamental do Cálculo (TFC)
   \item Teorema do valor médio para integrais (TVM)
  \end{compactitem}
\end{abstract}

\section*{Roteiro de aula}

  \begin{quotation}
    Nesta proposta de atividade de aprendizagem utilizaremos o recurso educacional (RE) interativo ``Integral de Riemann'' (disponível \href{http://cepa-usp.github.io/AI-0170/}{aqui}), Para instruções sobre como utilizar o OED, veja \href{http://www.youtube.com/watch?v=PJlPleMYuG4&t=22}{este tutorial}.
  \end{quotation}
  
  \subsection*{Objetivos}
  
    Ao final desta atividade, espera-se que os alunos:
    \begin{compactitem}
      \item Compreendam o significado o termo \emph{soma de Riemann}.
      \item Compreendam a importância do \emph{teorema fundamental do Cálculo}.
      \item Compreendam a relação entre os dois conceitos acima.
    \end{compactitem}

  \subsection*{Desenvolvimento da aula}
  
    Nesta proposta de atividade, construiremos uma soma de Riemann da \emph{função afim} $f(x) = 2x + 1$, cuja área sob ela pode ser calculada com base em argumentos geométricos (áreas do triângulo, do retângulo e do trapézio), conhecidos do aluno. Deste modo e utilizando implicitamente o \emph{teorema do valor médio} para integrais, poderemos construir uma soma de Riemann com poucos elementos de área e mostrar como o resultado da soma equipara-se ao cálculo analítico da integral definida, evidenciando assim o \emph{teorema fundamental do Cálculo}.
    
    Neste processo, é importante que os alunos sejam estimulados a participar, propondo o próximo passo na solução do problema que queremos resolver, a saber: encontrar o valor da área entre $f$ e o eixo $0x$ (a ``área sob $f$''), no intervalo $[a,b]$.
    
    Vejamos como fazer isso, passo-a-passo:
  
    \newcounter{steps}
    \begin{list}{\arabic{steps}.}{
	\setlength\leftmargin{0cm}%
	\setlength\itemindent\parindent%
	\setlength\listparindent{\parindent}%
        \setlength\labelwidth{1.5em}%
        \setlength\labelsep{0.5em}%
        \refstepcounter{steps}%
        \usecounter{steps}%
    }
  
      \item Utilize o RE para desenhar o gráfico da \emph{função afim} $f(x) = 2x + 1$. Para isso, escolha esta função na tela 1 do software, no grupo ``funções polinomiais'', e avance para a tela 2 (figura~\ref{fig:f}).
            
      \begin{figure}
	\begin{minipage}[t]{0.49\textwidth}
	  \includegraphics[width=\textwidth]{f.png}	  
	\end{minipage}
	\hfill
	\begin{minipage}[t]{0.49\textwidth}
	  \includegraphics[width=\textwidth]{grafico-f.png}
	\end{minipage}
	\caption{selecione a função $f(x) = 2x + 1$ na tela 1 (à esquerda) e avance para a tela 2, onde seu gráfico será automaticamente desenhado.}
	\label{fig:f}
      \end{figure}
      
      \item Escolha o intervalo de integração $[a,b]=[0,3]$: na tela 2, arraste o ponto $a$ para a abscissa $x = 0$ e o ponto $b$, para $x = 3$. Feito isso, avance para a tela 3.
      
      \item Escolha a opção ``soma personalizada'' na tela 3, o que lhe dará mais flexibilidade para construir a soma de Riemann. Avance para a tela 4.
      
      \item \label{step:area} Na tela 4 haverá, inicialmente, apenas os limites inferior ($a$) e superior ($b$) de integração. Questione os alunos sobre como calcular a área sob o gráfico de $f(x) = 2x + 1$ no intervalo escolhido. Espera-se que eles proponham utilizar a fórmula da área do trapézio de altura $b - a = 3$ e bases $f(a) = 1$ e $f(b) = 7$. Neste caso a área será
      \begin{equation*}
	\text{Área sob $f$} = \frac{1}{2}\left[f(a) + f(b)\right](b - a).
      \end{equation*}
      
      Alternativamente, os alunos podem sugerir somar a área do retângulo de lados $b - a$ e $f(a)$ com a área do triângulo de base $b - a$ e altura $f(b) - f(a)$. Neste caso, a área será dada por
      \begin{equation*}
       \text{Área sob $f$} = (b - a)f(a) + \frac{1}{2}(b - a)\left[f(b) - f(a)\right].
      \end{equation*}
      
      Obviamente as duas expressões resultam no mesmo valor: 12.

      \item Após determinar a área sob $f$ no intervalo $[a,b]$, questionar os alunos sobre como obter um valor \emph{aproximado} dessa área utilizando apenas retângulos. Espera-se que eles sugiram a inserção de vários retângulos abaixo do gráfico de $f$ (soma inferior) ou acima dele (soma superior), no intervalo de integração.
      
      Provavelmente os alunos questionarão o porquê dessa necessidade, haja vista que as fórmulas da Geometria Euclidiana Plana resolvem o problema, como visto no passo anterior. Se isso ocorrer, questione-os sobre como calcular áreas de regiões curvas, já direcionando a discussão para o objetivo da aula: somas de Riemann e o teorema fundamental do Cálculo.
    
      \item Após a discussão acima, ainda na tela 4, crie uma partição do intervalo $[a,b]$. Ou seja, escolha um conjunto de pontos sobre o exiso das abscissas que, juntamente com os pontos $a$ e $b$, dividem o intervalo de integração em $n = 5$ (ou mais) subintervalos. 
      
      Para criar um ponto da partição, clique com o mouse sobre o eixo das abscissas, na posição $x_i \in ]a,b[$ em que se quer criar o ponto. Em seguida, pressione o botão $+$ (mais), na barra de ferramentas, à esquerda da tela. Alternativamente, \emph{posicione} o mouse sobre o ponto desejado e pressione a tecla \textit{p}. Para remover este ponto, clique com o mouse sobre ele para selecioná-lo e pressione o botão $-$ (menos), ou pressione ``delete'' no teclado.
      
      Repita esse procedimento mais algumas vezes, procurando escolher os pontos $x_i$ de modo que a distância entre eles ($\Delta x_i$) varie, como ilustrado na figura \ref{fig:particao}. Feito isso, avance para a tela 5.
      
      \begin{figure}
	\centering
	\includegraphics[width=0.5\textwidth]{particao.png}
	\caption{exemplo de partição do intervalo $[a,b] = [0,3]$. Note como os pontos $x_i$ da partição foram escolhidos de modo que a distância entre eles varia. Isso é importante para mostrar a arbitrariedade inerente à construção de uma soma de Riemann.}
	\label{fig:particao}
      \end{figure}

      \item Na tela 5, construa os elementos de área, isto é, os retângulos de base $\Delta x_i = x_{i+1} - x_i$ e altura $f(\xi_i)$, onde $\xi_i$ é um número \emph{arbitrário} do subintervalo $[x_i, x_{i+1}]$ e $i = 0,1,\ldots,n-1$. Note que a área do $i$-ésimo elemento de área é igual a $f(\xi_i)\Delta x_i$.
      
      Para criar um elemento de área na ferramenta, clique no gráfico de $f$ em algum $\xi_i \in [x_i,x_{i+1}]$. Ao fazer isso, aparecerá o símbolo $\times$ sobre o gráfico de $f$, destacando o ponto $\left(\xi_i, f(\xi_i)\right)$. Neste momento, pressione o botão $+$ (mais) ou a tecla \textit{p} para que o elemento de área (um retângulo) seja automaticamente desenhado na tela. Se você repetir esse procedimento para um $\xi_i$ diferente do mesmo subintervalo, um novo elemento de área será criado no lugar do anterior. Para remover um elemento de área, clique sobre ele e pressione o botão $-$ (menos) ou a tecla ``delete''.
      
      Repita o procedimento acima para cada subintervalo, \emph{sem se preocupar com o critério de escolha de $\xi_i$}, pois um dos objetivos desta atividade é que esse ajuste seja feito manualmente.
      
      \item Ainda na tela 5, com todos os elementos de área presentes, questione os alunos sobre se a área encontrada, por meio dos retângulos, é uma boa aproximação. Note que você pode comparar a área exata ($=12$), calculada no passo \ref{step:area}, com a soma das áreas dos elementos de área, que é exibida no canto superior direito da tela.
      
      \item Motive os alunos a melhorar a aproximação. Para isso, chame a atenção para o fato de que em cada subintervalo há áreas ``sobrando'' (à esquerda de $\xi_i$) e ``faltando'' (à direita), e que a escolha de $\xi_i$ é arbitrária. Espera-se que os alunos proponham escolher $\xi_i$ de modo que, em cada subintervalo, o excesso de área à esquerda compense a falta dela à direita.
      
      Para fazer isso com o software, mova o mouse pelo subintervalo enquanto pressiona a tecla \textit{p} várias vezes, reconstruindo assim o elemento de área para vários $\xi_i$. Faça isso até que, \textbf{visualmente}, o excesso de área à esquerda de $\xi_i$ compense a falta à direita.\footnote{\textbf{Dica:} ao clicar sobre um elemento de área, aparecem duas retas verticais, uma à esquerda e outra à direita dele, que podem servir de guias para avaliar as áreas que faltam e que sobram.} Neste caso, $f(\xi_i)$ será o \emph{valor médio} de $f$ no subintervalo $[x_i, x_{i+1}]$.
      
      \paragraph{Observação:} quando escolhemos os $\xi_i$ de modo que $f(\xi_i)$ seja mínimo em cada subintervalo $[x_i,x_{i+1}]$, ao somar os elementos de área obtemos a chamada ``soma inferior'', que subestima a área sob $f$. Por outro lado, quando escolhemos $\xi_i$ de modo $f(\xi_i)$ seja máximo, obtemos a ``soma superior'', na qual a área é superestimada.
      
      \item Ajuste cada um dos $\xi_i$ conforme o critério do passo anterior. No final, compare novamente a soma das áreas dos elementos de área, no canto superior direito da tela, com a área obtida com argumentos geométricos (passo \ref{step:area}).
      
      Avance para a tela 6, tomando o cuidado de verificar se a quantidade de elementos de área desenhados é igual à quantidade $n$ de subintervalos: apenas nesta condição o software permitirá avançar.
      
      \item Na tela 6 o RE exibirá o gráfico da primitiva (ou anti-derivada) de $f$, a função $F$, sobreposta à construção realizada até aqui. Se quiser ver a expressão analítica dela, clique sobre $F$ para selecioná-la e pressione o botão ? (ajuda), no canto inferior esquerdo da tela. Nessa proposta, $F(x) = x^2/2 + x + C$, onde $C$ é a \emph{constante de integração}.
      
      \item Arraste $F$ para cima e para baixo, o que corresponde a variar $C$ (note que o valor da constante de integração varia conforme você faz isso: consulte o valor pressionando novamente o botão de ajuda, com a função $F$ selecionada). Argumente que essa ação não afeta a relação entre $f$ e $F$, isto é, $f = dF/dx$, pois a derivada de uma constante é nula.
      
      Essa etapa da atividade mostra que, ao integrar uma função, sua primitiva será, na verdade, uma família de funções, pois para cada valor de $C$, tem-se uma nova função.
      
      \item \label{step:C} Arraste $F$ para baixo, até que seu vértice aproxime-se de $y = -5$ (neste caso, $C \approx -4,7$). O intuito desse passo é apenas facilitar o passo seguinte.
      
      \item Obtenha o valor de $F(a)$: clique próximo do ponto $\left(a,F(a)\right)$ para destacá-lo no plano cartesiano, como ilustrado na figura \ref{fig:F(a)}. Em seguida, pressione o botão ? (ajuda) para ler o valor de $F(a)$ na janela de ajuda.
      
      \begin{figure}
	\begin{minipage}[t]{0.49\textwidth}
	  \includegraphics[width=\textwidth]{F(a).png}	  
	  \caption{selecione o ponto $\left(a,F(a)\right)$ e acesse a janela de ajuda para conhecer o valor de $F(a)$.}
	  \label{fig:F(a)}
	\end{minipage}
	\hfill
	\begin{minipage}[t]{0.49\textwidth}
	  \includegraphics[width=\textwidth]{linhas-guia.png}
	  \caption{utilize as linhas-guia verticais para selecionar o ponto $\left(b,F(b)\right)$.}
	  \label{fig:linhas-guia}
	\end{minipage}	
      \end{figure}
      
      \item Obtenha o valor de $F(b)$, como no passo anterior.
      
      Em alguns casos pode ser difícil acertar o clique sobre $\left(b,F(b)\right)$.\footnote{Especialmente se você não fez $C \approx 4,7$, como instruido no passo \ref{step:C}.} Para simplificar essa tarefa, selecione o elemento de área mais à direita para que o software desenhe as linhas-guia verticais nas laterais dele (fig~\ref{fig:linhas-guia}). Deste modo, basta clicar sobre a intersecção da linha vertical mais à direita com $F$ para obter $F(b)$.
      
      \item Calcule $F(b) - F(a)$ e compare o resultado com os valores da soma de Riemann, no canto superior direito, e da área obtida com argumentos geométricos (passo \ref{step:area}). O resultado desses três procedimentos deve ser o mesmo, igual a (ou aproximadamente) 12. Esta é uma evidência do \emph{teorema fundamental do Cálculo}, a saber:
      \begin{equation*}
       \text{Área sob $f$ em $[a,b]$} = \int_a^b f(x)\,dx = F(b) - F(a).
      \end{equation*}
      
      É fundamental chamar a atenção dos alunos para os dois últimos procedimentos: no primeiro, aproximamos a área sob $f$ pela soma da área de $n$ retângulos (\emph{matemática discreta}); no segundo, utilizamos a anti-derivada $F$ para calcular $F(b) - F(a)$ (\emph{matemática contínua}). A equivalência entre esses dois resultados é o teorema fundamental do Cálculo, que nos permite obter a \emph{integral definida de $f$ em $[a,b]$} (a área sob $f$) sem efetuar a laboriosa soma de Riemann.

    \end{list}


\section*{Nomenclatura}
\begin{compactitem}
 \item {\bfseries Partição} é o conjunto de pontos sobre o eixo das abscissas que dividem o intervalo de integração em $n$ subintervalos. Os limites inferior (representado por $a$) e superior ($b$) compõem o primeiro e último desses pontos, respectivamente.
 
 \item {\bfseries Teorema do valor médio para integrais (TVM):} se $f$ for uma função contínua em $[a,b]$, então existirá ao menos um $c \in \mathbb{R}$ tal que $\int_a^b f(x)\,dx = f(c)\cdot(b-a)$.
 
 \item A \emph{primitiva} de uma função $f$ é a a função $F$ tal que $f = \frac{dF}{dx} = \frac{d}{dx}\left[F(x) + C\right]$, onde $C$ é a \emph{constante de integração}.
 
 \item $x_i$ é o $i$-ésimo ponto da partição do interalo $[a,b]$.
 
 \item $\Delta x_i$ é a amplitude do $i$-ésimo subintervalo de integração.
 
 \item $\Delta a_i = f(\xi) \Delta x_i$ é o $i$-ésimo \emph{elemento de área}, um retângulo de base $\Delta x_i$ e altura $f(\xi_i)$, onde $\xi_i$ é um ponto do subintervalo de integração $[x_i, x_i + \Delta x_i]$.
 
 \item A \emph{soma de Riemann} $S_n$ é a soma de todos os elementos de área. Ou seja, $S_n = \sum_{i=1}{n} f(\xi_i) \Delta x_i$.
 
 \item A \emph{integral de Riemann} é o limite da soma de Riemann quando $n$ tende ao infinito. Ou seja, $I = \lim_{n \to \infty} S_n$.
 
 \item O \emph{teorema fundamental do Cálculo} (TFC) diz que, se $f$ for contínua em $[a,b]$, então $\int_a^b f(x)\,dx = F(b) - F(a)$, onde $F$ é qualquer antiderivada de $f$, isto é, uma função tal que $F' = f$.
\end{compactitem}

  \section{Créditos}
  
desenvolvido pelo CEPA (Centro de Ensino e Pesquisa Aplicada) da USP, em parceria com os professores Dr. Ivan Ramos Pagnossin, Ms. Rosilda dos Santos Morais e Ms. Shirlei Nabarrete Dezidério.  


\end{document}
