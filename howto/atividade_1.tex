\documentclass[a4paper,12pt]{scrartcl}
\usepackage[utf8]{inputenc}
\usepackage[T1]{fontenc}
\usepackage{gfsartemisia-euler}
\usepackage[brazil]{babel}
\usepackage{paralist}
\usepackage{amsmath}
\usepackage{amssymb}
\usepackage{geometry}
\usepackage{hyperref}
\usepackage[squaren]{SIunits}

\title{Proposta de atividade}
\subtitle{com o objeto educacional \emph{Integral de Riemann}}
\author{R. S. Morais, S. N. Dezidério, I. R. Pagnossin}

\begin{document}

\maketitle

\begin{abstract}
  Roteiro de aula para a introdução dos conceitos:
  \begin{compactitem}
   \item Soma de Riemann
   \item Teorema fundamental do Cálculo (TFC)
   \item Teorema do valor médio para integrais (TVM)
  \end{compactitem}
\end{abstract}

\section*{Roteiro de aula}

  \begin{quotation}
    Nesta proposta de atividade de aprendizagem utilizaremos o objeto educacional digital (OED) ``Integral de Riemann'' (disponível \href{http://cepa-usp.github.io/AI-0170/}{aqui}), desenvolvido pelo CEPA (Centro de Ensino e Pesquisa Aplicada) da USP, em parceria com os professores Dr. Ivan Ramos Pagnossin, Ms. Rosilda dos Santos Morais e Ms. Shirlei Nabarrete Dezidério. Para instruções sobre como utilizar o OED, veja \href{http://www.youtube.com/watch?v=PJlPleMYuG4&t=22}{este tutorial}.
  \end{quotation}

  \subsection*{Desenvolvimento da aula}

    Plotar o gráfico da \emph{função afim} $f(x) = 2x + 1$ no OED. Para isso, escolha esta função na tela 1. Em seguida, avance para a tela 2.
    
    Na tela 2, arraste os pontos $a$ e $b$ para definir o intervalo de integração $[a,b] = [0,3]$. Feito isso, avance para a tela 3.
    
    Na tela 3, opte por ``soma personalizada'' e avance para a tela 4.
    
    Na tela 4 haverá, inicialmente, apenas os limites inferior ($a$) e superior ($b$) de integração. Sobre essa região, questionar os alunos:
    
      \paragraph{Estratégia 1:} questionar os alunos sobre como proceder para calcular a área sob o gráfico da função $f(x) = 2x + 1$ no intervalo escolhido.
            
      A área da região no intervalo $[a,b] = [0,3]$ sob $f$ é um trapézio cuja base maior tem medida de \unit{7}{un} (unidades do gráfico), base menor de \unit{1}{un} e altura de \unit{3}{un}.
      
      \paragraph{Solução 1:} utilizar a fórmula da área do trapézio.
      \paragraph{Solução 2:} calcular a área do retângulo de dimensões \unit{3}{un} e \unit{1}{un} e somá-la à área do triângulo de base \unit{6}{un} e altura \unit{3}{un}
      
      \paragraph{Estratégia 2:} Questionar os alunos sobre como proceder para calcular a área desejada usando apenas a fórmula que calcula a área do retângulo. Espera-se que eles sugiram a inserção de retângulos abaixo do gráfico da $f$ (soma inferior) ou acima (soma superior) no intervalo $[a,b] = [0,3]$, a fim de obter a área desejada, ou uma área bem próxima dela. Provavelmente os alunos questionarão o porquê dessa necessidade, quando se conhece as fórmulas da Geometria Euclidiana Plana, vista acima (estratégia 1), que são imediatas. Se isso ocorrer, questioná-los sobre como calcular áreas de regiões curvas, já direcionando a discussão para o objetivo da aula: somas de Riemann e teorema fundamental do Cálculo.
      
    Após a discussão acima, ainda na tela 4, escolher a partição, isto é, um conjunto de pontos sobre o exiso das abscissas que, juntamente com os pontos $a$ e $b$, dividem o intervalo de integração em $n$ subintervalos. Procure escolher os pontos $x_i$ de modo que a distância entre eles ($\Delta x_i$) varie.
    
    Ao clicar no eixo das abscissas, o sinal de $+$ estará ativo no canto esquerdo da tela. Clicar nele para marcar o ponto. Repetir isso para todos os pontos e, em seguida, avançar para a tela 5.
    
    Na tela 5, clicar no gráfico de $f$, no intervalo $[a,b] = [0,3]$, para marcar as alturas dos retângulos. Ao clicar num determinado ponto da função (no intervalo de integração), você estará marcando as alturas $f(\xi_i)$ dos retângulos. Após clicar no gráfico de $f$, o sinal $+$ estará ativo no canto esquerdo da tela. Ao clicar sobre ele, o OED irá desenhar o retântulo. Faça isso de modo que haja um retângulo (elemento de área) para cada subintervalo de integração.
    
    Escolher a altura aleatoriamente sem se preocupar com o \emph{valor médio}, pois um dos objetivos da atividade é que esse ajuste seja feito manualmente.
    
    Questionar os alunos sobre se a área encontrada, por meio dos retângulos, é boa. Nessa etapa temos um elemento importante.

    Ainda na tela 5, vê-se nos retângulos que $\Delta x_i$ varia de um retângulo para outro. De acordo com a altura escolhida, vê-se que há áreas em excesso e em falta. Deve-se buscar por uma altura $f(\xi)$ do retângulo tal que as áreas em excesso (ou a falta) sejam mínimas ou se anulem. Isto é, uma compensa a outra. Para isso, colocar o cursor próximo do valor médio e pressionar a tecla $p$ em seu teclado, buscando a melhor altura. Se a altura do retângulo estiver bem próxima do valor médio, a área do retângulo será bem próxima da área procurada.
    
    Deve-se observar o valor da soma de Riemann, no canto superior direito da tela, bem como o número $n$ de retângulos. À medida que se procura pelo valor médio, nota-se que a área se aproxima cada vez mais da área procurada (calculada anteriormente, na estratégia 1). Essa condição será verificada na integral, que será mostrada na tela 6, pois quanto melhor for a escolha da altura dos retângulso, melhor será essa área.
    
    Vá para a tela 6: ajuste a constante de integração.
    
    Desejando conhecer essa função, ou sua expressão analítica, basta clicar sobre ela e, em seguida, em ?, que está no canto esquerdo da tela. Na janela aberta, no campo ``valor'', você conhecerá a expressão analítica da $F(x)$, que, nesse caso, é a $F(x) = x^2/2 + x + C$, ou seja, a função primitiva de $f(x) = 2x + 1$, ou ainda, a anti-derivada de $f$.
    
    Arraste a função primitiva para cima e para baixo na direção do eixo $0y$. Ao clicar no ? verifica-se o valor da constante de integração, que varia a cada movimento que se faz com a $F(x)$. Para cada ponto, uma nova constante $C$ aparecerá no campo ``valor'' (logo após a expressão da função). Essa etapa da atividade mostra que, ao integrar uma função, sua primitiva será $F(x) + C$, que é, na verdade, uma família de funções, pois para cada valor de $C$, tem-se uma nova função.
    
    Desde o início da atividade, desejou-se calcular a área sob a $f$ no intervalo de integração $[a,b] = [0,3]$, que é o mesmo que calcular a integral de $a$ até $b$. Viu-se que essa área calculada pela soma de Riemann era de, aproximadamente, \unit{12}{\squaren un}. Agora, na tela 6, pode-se ver a antiderivada de $f$ e, inclusive, calcular a integral, pelo teorema fundamental do Clálculo, confirmando o que já foi verificado pela soma de Riemann.
    
    O que se quer calcular é a integral de $a$ até $b$.
    
    A partir da primitiva, pode-se: clicar no elemento de área superior que o OED alongará os lados deste elemento, interceptando $F(x)$. Dessa forma, você encontrará $F(b)$ e poderá verificar seu valor, clicando no botão ? quando $F(b)$ estiver selecionado. Faça o mesmo no elemento de área inferior. Verifique o valor de $F(a)$ e calcule $F(b) - F(a)$. Essa diferença é a \emph{integral definida} de $a$ até $b$, pelo teorema fundamental do Cálculo, no intervalo $[a,b]$, que deve ser igual ou muito próxima da área calculada pela soma de Riemann.
    
    A integral definida de $a$ até $b$ pode ser calculada com o botão ?.
    
    Outra forma de calcular, também pelo teorema fundamental do Cálculo, mas de forma mais imediata, é deslocando a primitiva sobre o eixo $0y$, de tal modo que no ponto $a$ ela seja igual a zero ($F(a) = 0$). Assim, $F(b)$ será a área procurada.
    
    Em ambos os casos, a constante de integração é irrelevante: imagine que $F(x) = G(x) + C$. Então, $F(b) - F(a) = G(b) + C - G(a) - C = G(b) - G(a)$.
    
    Deve-se chamar a atenção dos alunos nessa etapa, no que se refere à passagem \emph{matemática discreta}, utilizada na soma de Riemann, com a inserção de $n$ retângulos, para a \emph{matemática contínua}, pois, nessa última, quando o número de retângulso $n$ tende ao infinido ($n \to \infty$), as áreas em excesso ou em falta tendem a zero. Esse é sem dúvida o nosso objetivo maior da aula. Os alunos precisam perceber que a soma de Riemann e o teorema fundamental do Cálculo respondem à pergunta levantada no início da atividade: qual é a área sob $f$?


\section*{Nomenclatura}
\begin{compactitem}
 \item {\bfseries Partição} é o conjunto de pontos sobre o eixo das abscissas que dividem o intervalo de integração em $n$ subintervalos. Os limites inferior (representado por $a$) e superior ($b$) compõem o primeiro e último desses pontos, respectivamente.
 
 \item {\bfseries Teorema do valor médio para integrais (TVM):} se $f$ for uma função contínua em $[a,b]$, então existirá ao menos um $c \in \mathbb{R}$ tal que $\int_a^b f(x)\,dx = f(c)\cdot(b-a)$.
 
 \item A \emph{primitiva} de uma função $f$ é a a função $F$ tal que $f = \frac{dF}{dx} = \frac{d}{dx}\left[F(x) + C\right]$, onde $C$ é a \emph{constante de integração}.
 
 \item $x_i$ é o $i$-ésimo ponto da partição do interalo $[a,b]$.
 
 \item $\Delta x_i$ é a amplitude do $i$-ésimo subintervalo de integração.
 
 \item $\Delta a_i = f(\xi) \Delta x_i$ é o $i$-ésimo \emph{elemento de área}, um retângulo de base $\Delta x_i$ e altura $f(\xi_i)$, onde $\xi_i$ é um ponto do subintervalo de integração $[x_i, x_i + \Delta x_i]$.
 
 \item A \emph{soma de Riemann} $S_n$ é a soma de todos os elementos de área. Ou seja, $S_n = \sum_{i=1}{n} f(\xi_i) \Delta x_i$.
 
 \item A \emph{integral de Riemann} é o limite da soma de Riemann quando $n$ tende ao infinito. Ou seja, $I = \lim_{n \to \infty} S_n$.
 
 \item O \emph{teorema fundamental do Cálculo} (TFC) diz que, se $f$ for contínua em $[a,b]$, então $\int_a^b f(x)\,dx = F(b) - F(a)$, onde $F$ é qualquer antiderivada de $f$, isto é, uma função tal que $F' = f$.
\end{compactitem}


\end{document}
